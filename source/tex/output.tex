\HlavickaUvod
\pdfbookmark{Úvod}{Uvod} \chapter*{Úvod}
\addcontentsline{toc}{chapter}{Úvod}

\begin{itemize}

\item
  motivace práce: současný stav strojově zpracovatelných prostorových
  dat pořizovaných a poskytovaných státní správou, možnosti jejich
  využití
\end{itemize}

\cleardoublepage
\HlavickaKapitoly
\setcounter{page}{1} \pagenumbering{arabic} \chapter{Cíl práce}

Práce se zaměřuje na ETL procesy spojené se zpracováním vybraných
datových sad poskytovaných státní správou České a Slovenské republiky.
Výměnné formáty státní správy jsou často poplatné nástrojům, které
konkrétní úřady používají při vytváření a správě dat, jejich převod do
prostředí open source GIS nástrojů nemusí být zcela triviální
záležitostí. Zároveň je však ETL proces naprosto nezbytným krokem k
dalšímu využití těchto dat, neboť úřady státní správy často používají
proprietární řešení, které další subjekty dispozici nemají.

Základním cílem práce je tedy představit komplexní řešení pro zpracování
souboru geografických informací (SGI) výměnného formátu katastru (VFK)
poskytovaného Českým úřadem zeměměřickým a katastrálním (ČÚZK), dále pak
formátu lesních hospodářských celků (LHC) poskytovaných Úřadem pro
hospodářskou úpravu lesa (ÚHÚL) a rovněž výměnného formátu souborů
geodetických informací (VGI) katastru nemovitostí poskytovaného
Geodetickým a kartografickým ústavem Bratislava (GKÚ).

Základním předpokladem k funkční implementaci těchto řešení je
samozřejmě dokonalé pochopení zmíněných formátů. Jejich popis a
identifikace možných úskalí je tedy rovněž jedním z cílů práce.

Všechny popisované postupy jsou implementovány pomocí open source
nástrojů. Funkční ETL procesy postavené na těchto nástrojích jsou
obchodním tajemstvím společnosti CleverMaps, a.s., proto nejsou veřejnou
součástí práce.
\chapter{Prostorová data státní správy}

\begin{itemize}

\item
  popsat stav, východiska (e-government?)
\item
  formáty (podrobněji v jednotlivých podkapitolách)
\item
  dostupnost dat (žádosti)
\end{itemize}
\chapter{ETL procesy}

\begin{itemize}

\item
  co to je
\item
  proč se to používá
\item
  best practices (?)
\item
  co z toho lze uplatnit při zpracování vybraných dat
\end{itemize}
\chapter{Implementace ETL procesů}

\begin{itemize}

\item
  popis workflow pro jednotlivé datové sady
\end{itemize}
\chapter{Popis výsledků a jejich využití}
\chapter*{Závěr}
