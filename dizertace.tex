
%%%%%%%%%%%%%%%%%%%%%%%%%%%%%%%%%%%%%%%%%%%%%%%%%%%%%%%%%%%%%%%%%%%%%%%%%%%%%%%%%%%%%%%%%%%%%%%%%%%%%%%
% Sablona Bc+Mgr+RNDr (CZ) pro PriF MU                                                              %%%
% Autor: Petr Zemanek (zemanekp@math.muni.cz)                                                       %%%
% Pripominky, dotazy, namety smerujte na diskusni forum: https://is.muni.cz/auth/df/sablona-prif    %%%
% Typeset in LaTeX-2e                                                                               %%%
% Verze: 1.9 (4. brezna 2016)                                                                       %%%
%%%%%%%%%%%%%%%%%%%%%%%%%%%%%%%%%%%%%%%%%%%%%%%%%%%%%%%%%%%%%%%%%%%%%%%%%%%%%%%%%%%%%%%%%%%%%%%%%%%%%%%

\documentclass[12pt,a4paper,oneside,final]{book}

\usepackage[utf8]{inputenc}
\usepackage[IL2]{fontenc}
\usepackage[czech]{babel}
\usepackage{longtable,lipsum}
\usepackage{amsmath,amssymb,amsthm}
\usepackage[RNDr,Barevne]{sci.muni.thesis}

\NazevUstavu{Geografický ústav}{Department of Geography}
\RokOdevzdaniPrace{rok}
\AkademickyRok{rok/rok}
\Autor{Michal Zimmermann}{Michal Zimmermann}
\NazevPrace{Název práce na titulní list}{Název práce}{Title of Thesis}
\StudijniProgram{Geografie}{Geography}
\StudijniObor{Kartografie, geoinformatika a dálkový průzkum Země}{Cartography, Geoinformatics and Remote Sensing}
\PocetStran{??\,$+$\,??}
\KlicovaSlova{ETL, transformace, prostorová data}{Keyword; Keyword; Keyword; Keyword; Keyword; Keyword; Keyword;
Keyword; Keyword}

\Abstrakty%
{V této bakalářské/diplomové/rigorózní práci se věnujeme ...}%
{In this thesis we study ...}

\TextPodekovani%
{Na tomto místě bych chtěl(-a) poděkovat ...}

\TextProhlaseni%
{Prohlašuji, že jsem svoji bakalářskou/diplomovou/rigorózní práci vypracoval(-a) samostatně s~využitím informačních
zdrojů, které jsou v práci citovány.}

\DatumProhlaseni{xx. měsíce 20xx}
\makeindex

\begin{document}
\VytvorPovinneStranyRigorozniPrace
\AbstraktyNaJedneStrane
% \AbstraktyNaDvouStranach
\SemVlozitZadani
% \VlozZadani{zadani.pdf}
\PodekovaniAProhlaseni
\pdfbookmark{Obsah}{Obsah}
\VytvorObsah
\cleardoublepage
% \HlavickaUvod
% \pdfbookmark{Úvod}{Uvod}
% \vloz{source/tex/01_Uvod}
% \cleardoublepage

\HlavickaZnaceni
% \pdfbookmark{Přehled použitého značení}{Prehled pouziteho znaceni}
% \vloz{source/tex/02_Znaceni}
\cleardoublepage

\renewcommand{\chaptermark}[1]{\markboth{\thechapter. #1}{}}
\renewcommand{\sectionmark}[1]{\markright{\thesection. #1}{}}
\addto\captionsczech{\renewcommand{\figurename}{Obr.}}
\addto\captionsczech{\renewcommand{\tablename}{Tab.}}
% \HlavickaKapitoly
% \vloz{source/tex/03_Kapitola_01}
% \cleardoublepage

% \vloz{source/tex/04_Kapitola_02}
% \cleardoublepage

\vloz{source/tex/output.tex}
\cleardoublepage

\HlavickaZaver
% \pdfbookmark{Závěr}{Zaver}
\vloz{source/tex/05_Zaver}
\cleardoublepage

\HlavickaPriloha
% \pdfbookmark{Příloha}{Priloha}
%\vloz{source/tex/06_Priloha}
\cleardoublepage

\renewcommand{\bibname}{Seznam použité literatury}
\HlavickaLiteratura
%\vloz{source/tex/07_Literatura}
\cleardoublepage

\renewcommand{\indexname}{Rejstřík}
\HlavickaRejstrik
\VytvorRejstrik
%%
%% pro vytvoreni rejstriku se spravny ceskym razenim pouzijte
%% csindex -d -h -k -z il2 nazev_souboru.idx
%%
%% nebo
%% texindy -I latex -L czech -M lang/czech/utf8 nazev_souboru.idx
%%

%%%%%%%%%%%%%%%%%%%%%%%%%%%%%%%%%%%%%%%%%%%%%%%%
%%%%%%%%%%% PRAZDNA STRANA NA ZAVER %%%%%%%%%%%%
%%%%%%%%%%%%%%%%%%%%%%%%%%%%%%%%%%%%%%%%%%%%%%%%

\newpage
\thispagestyle{empty}
\fancyhf{}
\newpage
\mbox{}

\end{document}
